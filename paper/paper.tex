\documentclass[fontsize=11pt]{scrartcl}
\usepackage[english]{babel}
\usepackage[utf8]{inputenc}				% special symbols
\usepackage{fancyhdr}					% headers and footers
\usepackage{verbatim}					% for comments
\usepackage{graphicx}					% images
\usepackage[font=footnotesize]{caption}	% smaller image label
\usepackage[usenames,dvipsnames,svgnames,table]{xcolor} 					% highlighting
\usepackage{hyperref}				    % for links (text or internet)
\usepackage{amsmath}	    			% maths
\usepackage{relsize}

\usepackage[hyperref=true,style=authoryear,backend=biber]{biblatex}
\usepackage{filecontents}
\usepackage{setspace}
\usepackage{fancyhdr}
\usepackage[ruled,vlined]{algorithm2e}

%\linespread{2}

% \pagestyle{headings}

\addbibresource{bibliography.bib}


\title{Real Time Fluid Simulation \\ \smaller using Smoothed-Particle Hydrodynamics and OpenGL}

\subtitle{Computer Graphics CS 488}

\author{Michael Berg \\ Matthias Untergassmair}


\pagestyle{fancy}
\rhead{\thepage}




\begin{document}


\maketitle

\abstract{
	This paper talks about the implementation of a real time Fluid simulation displayed in OpenGL \dots blaaaaaaaaaaaaa
}

\clearpage

\section{Introduction}
%
Hello World! This is how you can cite a paper \cite{realtimerendering} Let's go to a new line \\
Let's go to a new paragraph \par\medskip
Let's write a formula
\begin{align*}
	\sum_{i=0}^N \int_{-\infty}^\infty e^{-\frac{1}{2}x^2} \; dx
\end{align*}
%
It's not that hard ;)

\section{Mathematical Background}
%
\begin{align}
	a_i^n = \frac{F_i^n}{m_i} = \dots
	\label{math:acceleration}
\end{align}

\section{The Algorithm}
%
In the following, we denote the position for the particle $i$ at time $t$ as $x_i^t$, its velocity as $v_i^t$ and its acceleration as $a_i^t$.
We omit the vector notation ($\boldsymbol{x},\boldsymbol{v},\boldsymbol{a}$) for these quantities, since the following equations are valid for the vectors as well as for each component individually.
\par\medskip
%
As suggested in \cite{generalizedwallboundary}, we use the \emph{Velocity-Verlet} time stepping scheme as follows:
%
\begin{algorithm}
	\caption{Single Timestep with Velocity Verlet Algorithm}
	\KwData{$x_i^t$, $v_i^{t - \frac{\Delta t}{2}}$, $a_i^t$, $\Delta t$}
	\KwResult{$x_i^{t + \Delta t}$, $v_i^{t + \frac{\Delta t}{2}}$, $a_i^{t + \Delta t}$}	
	$v_i^{t+\frac{\Delta t}{2}} = v_i^{t - \frac{\Delta t}{2}} + \Delta t a_i^t$ \;
	$x_i^{t+\Delta t} = x_i^t + \Delta t v_i^{t+\frac{\Delta t}{2}}$ \;	
	$a_i^{t + \Delta t} = a_i^{t + \Delta t} (x_i^{t + \Delta t}, m_i)$ from equation \ref{math:acceleration} \;
\end{algorithm}


\section*{Appendix: Code}



\clearpage
\nocite{*}
\printbibliography


\end{document}
