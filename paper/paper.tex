\documentclass{sigchi}
\usepackage[english]{babel}
\usepackage[utf8]{inputenc}				% special symbols
\usepackage{fancyhdr}					% headers and footers
\usepackage{verbatim}					% for comments
\usepackage{graphicx}					% images
\usepackage[font=footnotesize]{caption}	% smaller image label
\usepackage[usenames,dvipsnames,svgnames,table]{xcolor} 					% highlighting
\usepackage{amsmath}	    			% maths
\usepackage{relsize}

\usepackage[hyperref=true,style=sigchi,backend=biber,doititles]{biblatex}
\usepackage{filecontents}
\usepackage{setspace}
\usepackage{fancyhdr}
\usepackage[ruled,vlined]{algorithm2e}

\usepackage{balance}  % to better equalize the last page
\usepackage{times}    % comment if you want LaTeX's default font
\usepackage{url}      % llt: nicely formatted URLs

\usepackage{listings}
\usepackage{placeins}

\lstset{language=C++,
    basicstyle=\footnotesize\ttfamily,
    keywordstyle=\bfseries\color{blue},%
    showstringspaces=false,
    breaklines=true,%
    morekeywords={include, printf},
	backgroundcolor=\color{lightgray},
    identifierstyle=\color{black},%
    stringstyle=\color{DarkOrchid},
    commentstyle=\color{Green},%
    numbers=left,
    numberstyle={\tiny \color{black}},
    numbersep=9pt,
    emph=[1]{for,if,else,switch,case,while,return},emphstyle=[1]\color{red},
	stepnumber=5,
	tabsize=2
}




% llt: Define a global style for URLs, rather that the default one
\makeatletter
\def\url@leostyle{%
  \@ifundefined{selectfont}{\def\UrlFont{\sf}}{\def\UrlFont{\small\bf\ttfamily}}}
\makeatother
\urlstyle{leo}


% To make various LaTeX processors do the right thing with page size.
\def\pprw{8.5in}
\def\pprh{11in}
\special{papersize=\pprw,\pprh}
\setlength{\paperwidth}{\pprw}
\setlength{\paperheight}{\pprh}
\setlength{\pdfpagewidth}{\pprw}
\setlength{\pdfpageheight}{\pprh}

% Make sure hyperref comes last of your loaded packages, 
% to give it a fighting chance of not being over-written, 
% since its job is to redefine many LaTeX commands.
\usepackage[pdftex]{hyperref}
\hypersetup{
pdftitle={SIGCHI Conference Proceedings Format},
pdfauthor={LaTeX},
pdfkeywords={SIGCHI, proceedings, archival format},
bookmarksnumbered,
pdfstartview={FitH},
colorlinks,
citecolor=black,
filecolor=black,
linkcolor=black,
urlcolor=black,
breaklinks=true,
}

% create a shortcut to typeset table headings
\newcommand\tabhead[1]{\small\textbf{#1}}




%\linespread{2}

% \pagestyle{headings}

\addbibresource{bibliography.bib}


\title{Real Time Fluid Simulation \\ \smaller using Smoothed-Particle Hydrodynamics and OpenGL}

\subtitle{Computer Graphics CS 488}

\numberofauthors{2}
\author{
  \alignauthor Matthias Untergassmair\\
    \email{munter2@uic.edu}\\
  \alignauthor Michael Berg\\
    \email{mberg4@uic.edu}\\
}


\begin{document}

\maketitle


\begin{abstract}
	This paper explores the field of Smooth Particle Hydrodynamics (SPH), starting at its 
beginnings as a tool to simulate astrophysical phenomena and following its evolution and 
implementation as a way to simulate fluids such as water. We will then give a brief 
mathematical background and ensuing algorithm of our SPH simulation followed by a detailed 
explanation of it.
\end{abstract}


\keywords{
	SPH, Smoothed Particle Hydrodynamics, OpenGL, Fluid Simulation, Real Time
}

Download code at
\begin{center}
	\url{https://github.com/munter2/RealTimeFluid}
\end{center}

\clearpage

\section{Introduction}
%
\hspace{6 pt} Smoothed Particle Hydrodynamics (SPH) successfully simulates fluids by breaking up a 
fluid body into individual parts, or particles. These particles together form a particle 
system that simulates various gravitational forces. Fluid movement is simulated in the 
system by moving particles around any particle moved, simulating a rippling, wave effect.
\\
\hspace*{6 pt} The ripple effect is created by first calculating which particles are surrounding 
a particle moved, and these surrounding particles are moved according to the movement 
of the first particle moved. But SPH wasn't originally intended to simulate liquid 
substances, but astrophysical phenomena.
%

\section{History}
%
\hspace{6 pt} Given all the different applications for Smoothed Particle Hydrodynamics (SPH),
it was first used to simulate interstellar phenomena. Conceived in 1977 by Gingold and 
Monaghan was an improvement to the Standard Finite Difference Method, which until their 
breakthrough, was the method to use to simulate astrophysical phenomena. They improved 
on this method by making "use of Lagrangian description of fluid flow which automatically 
focuses attention on fluid elements" \cite{sphastrophysics}. In this implementation, particles 
"move according to the Newtonian equations with forces due to the pressure gradient and 
other body forces: gravity, rotation and magnetic" \cite{sphastrophysics}.
\\
\hspace*{6 pt} The set of equations utilized in Gingold and Monaghan's SPH implementation 
is a density distribution calculated at each point
\begin{align}
	\rho_s(r) = \int W(r - r') \rho(r') dr'
\end{align}
where W is a function satisfying the condition
\begin{align}
\int W(r) dr = 1
\end{align}
\hspace*{6 pt} This density distribution equation is used to calculate the density of the 
particles surrounding the current particle the equation it's being applied to.
\hspace*{6 pt} A Gaussian function is used for the kernel function, with the form
\begin{align}
(\frac{1}{\pi h^2})^\frac{3}{2}  \exp{(\frac{-r^2}{h^2})}
\end{align}
and h is
\begin{align}
h = b(<r^2> - <r>^2)^\frac{1}{2}
\end{align}
and \textit{b} is adjustable.
\hspace*{6 pt} The gravitational potential of the particles is denoted by
\begin{align}
\phi = - G \int \frac{\rho_N(r')}{|r - r'|} dr'
\end{align}
with G being the gravitational constant. Substituting in eq. 1 turns the equation 
of gravitational potential into
\begin{align}
\phi = - \frac{GM}{N} \sum_{j=1}^N \int \frac{W(r - r')}{|r - r'|} dr'
\end{align}
To finish the equation, starting with
\begin{align}
\triangledown^2I_j = -4 \pi W(r - r_j)
\end{align}
substituting gravitational potential for I
\begin{align}
\triangledown \rho = - \frac{GM}{N} \sum_{j=1}^N \left \lbrace - \frac{4 \pi}{u_j^2} \int_{0}^{u_j} 
W(u) u^2 du \right \rbrace \triangledown u_j
\end{align}
where
\begin{align}
u_j = r - r_j
\end{align}
Adding in the Gaussian Function in eq. 3 for W yields
\begin{align}
\triangledown \phi = - \frac{GM}{N} \sum_{j=1}^N \frac{2}{u_j} \left ( \frac{f}{\pi} \right )^\frac{1}{2} 
\left \lbrack \exp{(-f u_j^2)} - \frac{1}{u_j} \int_{0}^{u_j} \exp{(-f u^2)} du \right \rbrack 
\triangledown u_j
\end{align}
This equation is the full solution to calculating the gravitational potential of all particles
involved in the simulation.
\\
\hspace*{6 pt} The result of Gingold and Monaghan's theory and implementation was a robust 
and extendable idea that could easily made more accurate "by increasing the number of particles 
and by using the devices known to improve Monte Carlo integration methods" \cite{sphastrophysics}.
%

\section{Smoothed Particle Hydrodynamics Overview}
%
\hspace{6 pt} Smoothed-particle hydrodynamics (SPH) method works by dividing the fluid into a set of 
referred to as particles. These particles have a spatial distance, or smoothing length, over which 
their properties are "smoothed" by a kernel function, such as a Gaussian function
\begin{align}
(\frac{1}{\pi h^2})^\frac{3}{2}  \exp{(\frac{-r^2}{h^2})}
\end{align}
This means that the physical quantity of any particle can be obtained by summing the 
relevant properties of all the particles which lie within the range of the kernel. For 
example, using Monaghan's popular cubic spline kernel the temperature at position r 
depends on the temperatures of all the particles within a radial distance 2h of r \cite{wikiarticle}.
\\
\hspace*{6 pt} The contributions of each particle to a property are weighted according to their distance 
from the particle of interest, and their density
A property for any one particle is calculated by discerning how many particles lay within
smoothing length of a given particle and what their respective properties are, and how 
dense that smoothing length is around the particle of interest. Mathematically, this is 
governed by a kernel function, like the Gaussian function given in eq. 1. This Gaussian
function gives large numbers for particles whose smoothing lengths are close to the 
particle of interest, and thus effect the calculation of a given particles properties, 
but as particles get far away, and their smoothing length becomes extremely small, 
although never reaches 0. This is due to intrinsic properties of the 
Gaussian function itself, but it is easy to account for this problem by testing for 
extremely small values close to zero, and disregarding those particles that are far away 
from the particle of interest as a result.
\\
\hspace*{6 pt} The equation for any quantity A at any point r is given by the equation
\begin{align}
A(r) = \sum_{j} m_j \frac{a_j}{p_j} W (|r - r_j|,h)
\end{align}
where m is the mass of particle j, A is the value of the quantity A for particle j, r 
denotes the position, p is the particle's density and W is a kernel function, such as the 
Gaussian function used by Gingold and Monaghan shown in eq. 1. The density of 
the particle can be computed in a variety of ways, such as the example equation
\begin{align}
p_i = p(r_i) = \sum_{j} m_j \frac{p_j}{p_j} W (|r_i - r_i|,h) = \sum_{j} m_j W(r_i - r_j,h)
\end{align}
Similarly, the spatial derivative of a quantity can be computed as follows
\begin{align}
\triangledown A(r) = \sum_{j} m_j \frac{a}{p_j} \triangledown W(|r - r_j|, h)
\end{align}
Although the size of the smoothing length can be fixed in both space and time, this does 
not take advantage of the full power of SPH. By assigning each particle its own smoothing 
length and allowing it to vary with time, the resolution of a simulation can be made to 
automatically adapt itself depending on local conditions. For example, in a very dense 
region where many particles are close together the smoothing length can be made relatively 
short, yielding high spatial resolution. Conversely, in low-density regions where 
individual particles are far apart and the resolution is low, the smoothing length can be 
increased, optimising the computation for the regions of interest. Combined with an 
equation of state and an integrator, SPH can simulate hydrodynamic flows efficiently \cite{wikiarticle}.
\\
\hspace*{6 pt} Given the principles of smoothing length, calculating properties based on 
the area around a particle, and increasing or decreasing a property given the density of 
particles in an area, we applied some of these concepts in our implementation of SPH.
%

\section{Mathematical Background}
%
\begin{align}
	a_i^n = \frac{F_i^n}{m_i} = \dots
	\label{math:acceleration}
\end{align}

\section{The Algorithm}
%
In the following, we denote the position for the particle $i$ at time $t$ as $x_i^t$, its velocity as $v_i^t$ and its acceleration as $a_i^t$.
We omit the vector notation ($\boldsymbol{x},\boldsymbol{v},\boldsymbol{a}$) for these quantities, since the following equations are valid for the vectors as well as for each component individually.
\par\medskip
%

\balance

\nocite{*}
\printbibliography

\clearpage
\begin{comment}
\clearpage

\section*{Appendix: Code}

\lstinputlisting[caption=sphModel.hpp]{../sphModel.hpp}
\lstinputlisting[caption=sphModel.cpp]{../sphModel.cpp}
\lstinputlisting[caption=Simulation.cpp]{../Simulation.cpp}
\lstinputlisting[caption=simulation.vsh]{../resources/simulation.vsh}
\lstinputlisting[caption=simulation.fsh]{../resources/simulation.fsh}

\end{comment}













\clearpage


Several resources point out the algorithms to be used for applying SPH to Fluid Dynamics, the algorithm is described particularly clearly in \cite{kennyerleben} and can be summarized as follows:
%
\FloatBarrier
\begin{algorithm}
	\caption{SPH simulation}
	\For{$i = 1, \dots, N_{\text{Particles}}$}{
		Compute the density of the $i$th particle
		\begin{align*}
			\rho_i = \sum_j m_j W(\| x_i - x_j \|, h)
		\end{align*}
		Compute the pressure of the $i$th particle
		\begin{align*}
			p_i = k (\rho_i - \rho_0)
		\end{align*}
		Compute the interaction forces acting on the $i$th particle, consisting of pressure forces, viscosity and gravity
		\begin{align*}
			f_i^p &= - \sum_j m_j \frac{p_i + p_j}{2 \rho_j} \nabla W(\| x_i - x_j \|, h) \\
			f_i^v &= \mu \sum_j m_j \frac{v_j + v_i}{\rho_j} \nabla^2 W(\| x_i - x_j \|, h) \\
			f_i^{\text{total}} &= f_i^p + f_i^v + g
		\end{align*}
		Solve the differential equation
		\begin{align}
			\begin{aligned}
				\dot{x}_i &= v_i \\
				\dot{v}_i &= \frac{f_i}{\rho_i}
			\end{aligned}
			\label{math:ode}
		\end{align}
	}
\end{algorithm}
\FloatBarrier
%
In our implementation, we use the Velocity Verlet (or Leap-Frog) scheme to solve the ODE \ref{math:ode}, as suggested in \cite{generalizedwallboundary}.
The scheme looks as follows:
%
\begin{algorithm}
	\caption{Single Timestep with Velocity Verlet Algorithm}
	\KwData{$x_i^t$, $v_i^{t - \frac{\Delta t}{2}}$, $a_i^t$, $\Delta t$}
	\KwResult{$x_i^{t + \Delta t}$, $v_i^{t + \frac{\Delta t}{2}}$, $a_i^{t + \Delta t}$}	
	$v_i^{t+\frac{\Delta t}{2}} = v_i^{t - \frac{\Delta t}{2}} + \Delta t a_i^t$ \;
	$x_i^{t+\Delta t} = x_i^t + \Delta t v_i^{t+\frac{\Delta t}{2}}$ \;	
	$a_i^{t + \Delta t} = a_i^{t + \Delta t} (x_i^{t + \Delta t}, m_i)$ from equation \ref{math:acceleration} \;
\end{algorithm}


\clearpage


As a kernel function we used the standard kernel, the \emph{cubic spline}
\begin{align*}
	W_3(r,h) := \frac{1}{\pi h^3}
	\begin{cases}
		1 - \frac{3}{2} \xi^2 + \frac{3}{4} \xi^3, &\text{for } 0 \le \xi < 1 \\
		\frac{1}{4} (2-\xi)^3, &\text{for } 1 \le \xi < 2 \\
		0, &\text{otherwise}
	\end{cases}
\end{align*}
%
where $\xi := \frac{r}{h}$. It can be seen clearly that the support of this function is $2h$. \\
The derivative is
%
\begin{align*}
	\nabla W_3(r,h) &= \frac{\partial}{\partial r} W_3(r,h) + \frac{\partial}{\partial h} W_3(r,h)
\end{align*}
%
\begin{itemize}
	\item for $0 \le \xi < 1$
		\begin{align*}
			\frac{\partial}{\partial r} W_3(r,h)|_{0 \le \xi < 1}
			&= \frac{1}{\pi h^3} \frac{1}{h} \left( 3 \xi + \frac{9}{4} \xi^2 \right)
		\end{align*}
		%
		\begin{align*}
			\frac{\partial}{\partial h} W_3(r,h)|_{0 \le \xi < 1}
			&= \frac{-3}{\pi h^4} \left( 1 - \frac{3}{2} \xi^2 + \frac{3}{4} \xi^3 \right) \\
			&\quad + \frac{1}{\pi h^3} \frac{1}{r} \left( (-2) \frac{3}{2} \xi^3 + (-3)\frac{3}{4} \xi^4 \right) \\
			&= - \frac{3}{\pi h^4} \left( 1 - \frac{3}{2} \xi^2 + \frac{3}{4} \xi^3 \right) \\
			&\quad + \frac{1}{\pi h^3} \frac{1}{r} \left( -3 \xi^3 -\frac{9}{4} \xi^4 \right)
		\end{align*}
		%
		\begin{align*}
			\Rightarrow \nabla W_3(r,h)|_{0 \le \xi < 1}
		&= \frac{1}{\pi h^4} \left( 3 \xi + \frac{9}{4} \xi^2 \right) \\
			&- \frac{3}{\pi h^4} \left( 1 - \frac{3}{2} \xi^2 + \frac{3}{4} \xi^3 \right) \\
			&\quad + \frac{1}{\pi h^3} \frac{1}{r} \left( -3 \xi^3 -\frac{9}{4} \xi^4 \right) \\
		&= \frac{3}{\pi h^4} \left( -1 + \xi + \frac{9}{4} \xi^2
			- \frac{3}{4} \xi^3 \right) \\
			&\quad - \frac{1}{\pi r h^3} \xi^3 \left( 3 + \frac{9}{4} \xi \right)
	\end{align*}
	\item for $1 \le \xi < 2$
		\begin{align*}
			\frac{\partial}{\partial r} W_3(r,h)|_{1 \le \xi < 2}
			&= \frac{1}{4} 3(2-\xi)^2 \cdot \left( -\frac{1}{h} \right) \\
			&= -\frac{3}{4h} (2-\xi)^2
		\end{align*}
		%
		\begin{align*}
			\frac{\partial}{\partial h} W_3(r,h)|_{1 \le \xi < 2}
			&= \frac{1}{4} 3(2-\xi)^2 \cdot \left( +\frac{\xi}{h} \right) \\
			&= \frac{3 \xi}{4 h} (2-\xi)^2
		\end{align*}
		\begin{align*}
			\Rightarrow \nabla W_3(r,h)|_{1 \le \xi < 2}
			&= -\frac{3}{4h} (2-\xi)^2
			+ \frac{3 \xi}{4 h} (2-\xi)^2 \\
			&= \frac{3}{4h} (2-\xi)^2 (\xi - 1)
		\end{align*}
	\item otherwise
		\begin{align*}
			\nabla W_3(r,h)|_{\text{otherwise}} = 0
		\end{align*}
\end{itemize}
%
The following two code snippets show how these formulas are implemented in our simulation:
\begin{lstlisting}[caption=Cubic Spline]
// Cubic spline function
inline float W3(float r, float h) {
	float W = 0;
	float zeta = r/h;
	if(0 <= zeta && zeta <= 2) {
		W = (zeta < 1 ? 
				1 - .75*zeta*zeta*(2+zeta) :
				.25*(2-zeta)*(2-zeta)*(2-zeta)
		);
	}
	W /= (M_PI*h*h*h);
	return W;
}
\end{lstlisting}
%
\begin{lstlisting}[caption=Derivative of Cubic Spline]
Hello World
\end{lstlisting}







\end{document}
